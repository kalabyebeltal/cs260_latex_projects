\documentclass[addpoints]{exam}

\usepackage{listings}
\usepackage{amsmath}
\usepackage{amsfonts}
\usepackage{mathtools}
\usepackage{tikz}
\usepackage{qtree}
\usepackage{tikz-qtree}
\usepackage{graphicx}
\usepackage{enumitem}
\usepackage[norelsize,noline,noend,linesnumbered]{algorithm2e}
\usetikzlibrary{arrows,automata}
\usepackage{hyperref}
\hypersetup{
    colorlinks=true,
    linkcolor=blue,
    filecolor=magenta,
    urlcolor=cyan,
}
\usepackage{mdframed}


\begin{document}

\title{CS260 - Written Assignment 1}
\author{Kal-Ab Yebeltal}
\date{Winter 2025-26}
\maketitle



\begin{questions}

%%%

\question For each of the following functions:

\begin{itemize}
\item Model its runtime using summation(s) (if needed).
\item Simplify to a closed form and give a tight upper bound on the runtime.
\end{itemize}

If the runtime is asymptotically different in different cases, briefly explain what the best and worst cases are and give a separate analysis for each case.

Algorithms have cases when something other than the input size can asymptotically affect the runtime. Linear search has a best case runtime of $O(1)$ when the target is at the beginning of the list and a worst case runtime of $O(n)$ when the target isn't in the list. Not all algorithms have cases; a function that merely sums all of the numbers in the list always runs in time $O(n)$.

\begin{parts}

%

\part[10] \texttt{foo}

\begin{verbatim}
/* a, b, c, d are all integers */
function foo(a, b, c, d) {
  if (a < b) {
    if (c < d) {
      if (a < c) {
        return a
      } else {
        return c
      }
    } else {
      if (a < d) {
        return a
      } else {
        return d
      }
    }
  } else {
    if (c < d) {
      if (b < c) {
        return b
      } else {
        return c
      }
    } else {
      if (b < d) {
        return b
      } else {
        return d
      }
    }
  }
}
\end{verbatim}

\textbf{Solution:}

$\sum_{i=1}^{3}c = 3c = \Theta(1)$

%

\part[10] \texttt{bar}

\begin{verbatim}
/* L is array of n integers */
function bar(L, n) {
  for (i = 0; i < n; i++) {
    if (L[i] == 7) {
      return true
    }
  }
  return false
}
\end{verbatim}

\textbf{Solution:}

Best Case:

$\sum_{i=1}^{1}c = c = \Theta(1)$

Worst Case:

$\sum_{i=1}^{n}c = cn = \Theta(n)$

Overall Case: $O(n)$

%

\part[10] \texttt{baz}

\begin{verbatim}
/* n is a positive integer */
function baz(n) {
  value = 0
  for (i = 0; i < n; i++) {
    for (j = 0; j < n; j++) {
      for (k = 0; k < n; k++) {
        for (l = 0; l < 8; l++) {
          value = value + i * (j + k) - l
        }
      }
    }
  }
  return value
}
\end{verbatim}

\textbf{Solution:}

$\sum_{l=0}^{7}c = 8c$

$\sum_{l=0}^{7} \sum_{k=0}^(n-1)c = \sum_{k=0}^(n-1)8c = 8cn$

$\sum_{j=0}^{n-1} \sum_{l=0}^{7} \sum_{k=0}^(n-1)c = \sum_{j=0}^(n-1)8cn = 8cn^2$

$\sum_{i=0}^{n-1}\sum_{j=0}^{n-1} \sum_{l=0}^{7} \sum_{k=0}^(n-1)c = \sum_{i=0}^(n-1)8cn^2 = 8cn^3 = \Theta(n^3)$

%

\part[10] \texttt{qux}

\begin{verbatim}
/* n is a positive integer */
function qux(n) {
  x = 2
  for (i = 0; i < n; i++) {
    x = x * 7
  }
  for (j = 0; j < 2*n; j++) {
    x = x / 3
  }
  return x
}
\end{verbatim}

\textbf{Solution:}

$\sum_{i=0}^{n-1}c = cn$
%

\part[10] \texttt{xyzzy}

\begin{verbatim}
/* L is array of n integers */
function xyzzy(L, n) {
  a = 0
  while (a < n) {
    b = 0
    x = 0
    while (b < n) {
      x = x + L[b]
      b++
    }
    L[a] = x
    a++
  }
  return L
}
\end{verbatim}


\textbf{Solution:}

$\sum_{b=0}^{n-1}c = cn$

$\sum_{a=0}^{n-1}\sum_{b=0}^{n-1}c = cn^2 = \Theta(n^2)$

%

\end{parts}

%%%

\question[30] Order the following functions from least to greatest growth rate, such that for any two functions $f$ and $g$, if $f$ comes before $g$, then $f(n) = o(g(n))$:

\begin{itemize}
\item $f_1(n) = n^n$
\item $f_2(n) = n^{0.00001}$
\item $f_3(n) = n^2$
\item $f_4(n) = \lg(n)$
\item $f_5(n) = 2^n$
\end{itemize}

\textbf{You must prove each ranking}, but you may assume the transitive property is true, which says that if $f(n) = o(g(n))$ and $g(n) = o(h(n))$, then $f(n) = o(h(n))$.

This means that if the functions were $f, g, h$ and you gave the order $g, h, f$, then you'd only need to prove that $g(n) = o(h(n))$ and $h(n)= o(f(n))$, as you may assume that this implies $g(n) = o(f(n))$.

\textbf{You must fully simplify your limit ratios to receive credit.} If you haven't simplified your limit ratio to 0, $\infty$, or a positive constant, you're not finished simplifying.

\ 

\ 

\ 

\ 

\ 

\textbf{Solution:}

1. $f_1(n)$ and $f_2(n)$:

\begin{align*} \lim_{n \to \infty} \frac{n^{0.00001}}{n^n} &= \lim_{n \to \infty} n^{0.00001 - n} \ = 0 \end{align*}

Therefore, $f_2(n) = o(f_1(n))$

\

\

2. Compare $f_1(n)$ and $f_3(n)$:

\begin{align*} \lim_{n \to \infty} \frac{n^2}{n^n} &= \lim_{n \to \infty} n^{2-n}  \ = 0 \end{align*}

Therefore, $f_3(n) = o(f_1(n))$

%%%

\question 

\begin{parts}

%

\part[5] Prove or disprove: $2^n = o(3^n)$
YOUR SOLUTION HERE

%

\part[5] Would your answer to the previous question remain the same for any pair of positive and constant exponent bases, e.g. 7 and 5, or 11 and 32? Explain your answer.
YOUR SOLUTION HERE

%

\end{parts}

%%%

\question 

\begin{parts}

%

\part[5] Prove or disprove: $2^n = o(2^{n+1})$
YOUR SOLUTION HERE

%

\part[5] Would your answer to the previous question remain the same for any constant amount of increase to the power, e.g. $2^{n+100}$? Explain your answer.
YOUR SOLUTION HERE

%

\end{parts}

%%%

\end{questions}

\end{document}
